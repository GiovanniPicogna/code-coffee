\documentclass{aa}
\usepackage{showyourwork}
\usepackage{graphicx}
\usepackage{txfonts}
\usepackage{listings}
\usepackage{subcaption}         % necessary for continued figures, example in section 3
                                % and appendix
\usepackage{lscape}             % to rotate a single page table, example in appendix.
                                % For landscape tables, see the longtable examples.
\usepackage{placeins}           % useful with \FloatBarrier, to keep 
                                % onecolumn floats from drifting to the next section

% -- Loading the code block package:
\setlength{\parindent}{8pt}
\usepackage{indentfirst}% -- Defining colors:
\definecolor{codegreen}{rgb}{0,0.6,0}
\definecolor{codegray}{rgb}{0.5,0.5,0.5}
\definecolor{codepurple}{rgb}{0.58,0,0.82}
\definecolor{backcolour}{rgb}{0.95,0.95,0.92}% Definig a custom style:
\definecolor{navyblue}{rgb}{0.0, 0.0, 0.5}
\lstdefinestyle{mystyle}{
    backgroundcolor=\color{backcolour},   
    commentstyle=\color{codepurple},
    keywordstyle=\color{navyblue},
    numberstyle=\tiny\color{codegray},
    stringstyle=\color{codepurple},
    basicstyle=\ttfamily\footnotesize\bfseries,
    breakatwhitespace=false,         
    breaklines=true,                 
    captionpos=t,                    
    keepspaces=true,                 
    numbers=left,                    
    numbersep=5pt,                  
    showspaces=false,                
    showstringspaces=false,
    showtabs=false,                  
    tabsize=2
}% -- Setting up the custom style:

\lstset{style=mystyle}

\begin{document}

\title{\showyourwork for open science}

\subtitle{Test cases}

\author{Giovanni Picogna}

\institute{Universit\"ats-Sternwarte, Ludwig-Maximilians-Universit\"at M\"unchen, Scheinerstr. 1, M\"unchen, 81679, Bayern, Germany\\
             \email{picogna@usm.lmu.de}}

\date{Received 14 July 2025 / Accepted 14 July 2025}

\abstract {} {I provide some easy to follow instructions to set-up your first \showyourwork project}
{This is not at all a complete guide, for which I point the readers at the well written documentation on https://show-your.work/}
{}{}{}

\keywords{open science}

\maketitle 

% Main body with filler text
\section{Installation}
\label{sec:intro}
\showyourwork requires Python <3.11, >=3.8 in the current development version. We can start by creating a conda environment with python 3.10
  \begin{lstlisting}[language=bash]
    conda create -n python310 python=3.10 anaconda
  \end{lstlisting}
activate it
  \begin{lstlisting}[language=bash]
    conda activate python310
  \end{lstlisting}
and install the development version from github
  \begin{lstlisting}[language=bash]
    pip install git+https://github.com/showyourwork/showyourwork
  \end{lstlisting}

\section{First paper}
\subsection{Local build}
In order to set up your first paper you need to run
  \begin{lstlisting}[language=bash]
    showyourwork setup ${GITHUB_USER}/${GITHUB-REPO}
  \end{lstlisting}
and create the related repo on github (follow the instruction on screen - you can add the caching support on Zenodo and the Overleaf connection also in a second step)
A working \texttt{environment.yml} for the conda environment is:
  \begin{lstlisting}[language=bash]
    channels:
      - conda-forge
      - defaults
    dependencies:
      - numpy=1.21
      - pip=25.1.1
      - python=3.10
      - pip:
        - matplotlib==3.10
  \end{lstlisting}
You can then build first your paper locally
  \begin{lstlisting}[language=bash]
    showyourwork build
  \end{lstlisting} 
and check the resulting ms.pdf file in the root directory for the final result.

\subsection{Remote build on github}
In order to build it on github with github actions you need first to include/substitute in the files \texttt{build.yml} and \texttt{build-pull-request.yml} under the directory \texttt{.github/workflows} the following lines:
  \begin{lstlisting}[language=xml]
    - name: Build the article PDF
      id: build
      with:
        showyourwork-spec: git+https://github.com/showyourwork/showyourwork
      uses: showyourwork/showyourwork-action@main
      env:
        SANDBOX_TOKEN: ${{ secrets.SANDBOX_TOKEN }}
        OVERLEAF_TOKEN: ${{ secrets.OVERLEAF_TOKEN }}
  \end{lstlisting}
then you can \texttt{git add} the changed files, commit it and push it to the remote repository.

\section{Adding a figure}
\subsection{Simple figure}
\showyourwork does not expect you to provide directly a figure (in fact it gives an error if you do).
It expects instead a script to generate the plot through the \texttt{script} command and, after running it, it looks for the figure obtained with it.
Fig.~\ref{fig:random_numbers} is an example. The script \texttt{randomnumbers.py} is placed in the \texttt{src/script/} directory and called in the first line of the figure definition. After that, the figure is included normally from the predefined path.
\begin{lstlisting}[language=TeX]
\begin{figure}
    \script{random_numbers.py}
    \begin{centering}
        \includegraphics[width=\linewidth]{figures/random_numbers.pdf}
        \caption{
            Plot showing a bunch of random numbers.
        }
        \label{fig:random_numbers}
    \end{centering}
\end{figure}
\end{lstlisting}
The script is generating and plotting some random numbers using the \texttt{numpy random} package 
\begin{lstlisting}[language=python]
import matplotlib.pyplot as plt
import numpy as np
import paths

# Generate some data
random_numbers = np.random.randn(100, 10)

# Plot and save
fig = plt.figure(figsize=(7, 6))
plt.plot(random_numbers, 'k.')
plt.xlabel("x")
plt.ylabel("y")
fig.savefig(paths.figures / "random_numbers.pdf", bbox_inches="tight", dpi=300)
\end{lstlisting}

\begin{figure}
    \script{random_numbers.py}
    \begin{centering}
        \includegraphics[width=\linewidth]{figures/random_numbers.pdf}
        \caption{
            Plot showing a bunch of random numbers.
        }
        \label{fig:random_numbers}
    \end{centering}
\end{figure}

\subsection{Static figure}
It is possible that some figures are not generated from a script, i.e. an hand drawn figure or a figure from another paper. In that case \texttt{showyouwork!} provides a static directory under \texttt{src/static} where any static content can be added. In this case the figure is defined normally in \LaTeX, but the location of the figure is always within the figures folder (\showyourwork is copying it under the wood from the static directory to the \LaTeX figures directory).
\begin{lstlisting}[language=TeX]
\begin{figure}
    \begin{centering}
        \includegraphics[width=\linewidth]{figures/usmlogo.pdf}
        \caption{
            USM Logo.
        }
        \label{fig:usm_logo}
    \end{centering}
\end{figure}
\end{lstlisting}

\begin{figure}
    \begin{centering}
        \includegraphics[width=\linewidth]{figures/usmlogo.pdf}
        \caption{
            USM Logo.
        }
        \label{fig:usm_logo}
    \end{centering}
\end{figure}

\section{Zenodo integration}

\subsection{Static datasets}
If your workflow depends on data that cannot be programmatically generated (e.g. data collected from a telescope), that data should be made available to anyone trying to reproduce your results. Instead of committing the dataset directly to the repository, one can archive it on an online open-access file-hosting service (like Zenodo, for which \showyourwork does all the communicating back-and-forth for you). All you need to do is specify the ID of the (public) archive and some information about the files your workflow needs in the \texttt{showyourwork.yml} config file.

\subsection{Dynamic datasets}
\showyourwork can cache the results of intermediate steps in your pipeline on Zenodo or Zenodo Sandbox.
This is useful for workflows that need running expensive simulations, etc., that a reader may not want to rerun.
It's also useful for builds on GitHub Actions, which has limited compute resources.

The way \showyourwork deals with these cases is to cache these lengthy computations on Zenodo alongside a record of all the inputs that went into generating the cached output. If, on subsequent runs of the workflow, the inputs remain unchanged, \showyourwork will simply download the cached results from Zenodo, maintaining the guarantee that the output you get follows deterministically from the given inputs.

\subsubsection{Set-up the Zenodo integration}
By default, caching takes place on Zenodo Sandbox, which is equal to Zenodo in all respects except that records on that service are temporary. This makes it perfect for caching intermediate results during the development cycle, where things are likely to change a lot and you may not want to assign a static, permanent DOI with any particular dataset.

\showyourwork needs access to an API token to communicate with Zenodo Sandbox.
You can generate it by clicking \href{https://sandbox.zenodo.org/account/settings/applications/tokens/new}{here}.
Name the token something informative and make sure to give it \texttt{deposit:actions} and \texttt{deposit:write} permissions.
Copy the token and store it somewhere secure.

Then, on your local computer, create an environment variable called \texttt{\$SANDBOX\_TOKEN} with value equal to the API token you just generated.
You can either do this manually: \texttt{export SANDBOX\_TOKEN=YYYYYY} or by adding that line to your shell config file (.bashrc, .zshrc, etc.) and re-starting your session.
In order for \showyourwork to have access to Zenodo Sandbox when running on GitHub Actions, you'll also have to provide this value as a secret with name \texttt{SANDBOX\_TOKEN}.

If you've done all that, the next time you create a new article repository using \texttt{showyourwork setup}, pass the \texttt{--cache} option and \showyourwork will automatically create a Zenodo Sandbox draft deposit which it will use to cache your intermediate results. Note that you can also manually create a draft deposit by running \texttt{showyourwork cache create} after you created your article repository.

\subsubsection{Intermediate results}

\begin{lstlisting}[language=TeX]
\begin{figure}
    \script{figure.py}
    \begin{centering}
        \includegraphics[width=\linewidth]{figures/figure.pdf}
        \caption{
            Plot showing the result of a DustPy simulation.
        }
        \label{fig:simulation_dustpy}
    \end{centering}
\end{figure}
\end{lstlisting}

\begin{figure}
    \script{figure.py}
    \begin{centering}
        \includegraphics[width=\linewidth]{figures/figure.pdf}
        \caption{
            Plot showing the result of a DustPy simulation.
        }
        \label{fig:simulation_dustpy}
    \end{centering}
\end{figure}

\section{Conclusions}

\begin{acknowledgements}
\end{acknowledgements}

\bibliography{bib} % your references Yourfile.bib

\end{document}
