\documentclass{aa}
\usepackage{showyourwork}
\usepackage{graphicx}
\usepackage{txfonts}
\usepackage{listings}
\usepackage{subcaption}         % necessary for continued figures, example in section 3
                                % and appendix
\usepackage{lscape}             % to rotate a single page table, example in appendix.
                                % For landscape tables, see the longtable examples.
\usepackage{placeins}           % useful with \FloatBarrier, to keep 
                                % onecolumn floats from drifting to the next section

% -- Loading the code block package:
\setlength{\parindent}{8pt}
\usepackage{indentfirst}% -- Defining colors:
\definecolor{codegreen}{rgb}{0,0.6,0}
\definecolor{codegray}{rgb}{0.5,0.5,0.5}
\definecolor{codepurple}{rgb}{0.58,0,0.82}
\definecolor{backcolour}{rgb}{0.95,0.95,0.92}% Definig a custom style:
\lstdefinestyle{mystyle}{
    backgroundcolor=\color{backcolour},   
    commentstyle=\color{codepurple},
    keywordstyle=\color{NavyBlue},
    numberstyle=\tiny\color{codegray},
    stringstyle=\color{codepurple},
    basicstyle=\ttfamily\footnotesize\bfseries,
    breakatwhitespace=false,         
    breaklines=true,                 
    captionpos=t,                    
    keepspaces=true,                 
    numbers=left,                    
    numbersep=5pt,                  
    showspaces=false,                
    showstringspaces=false,
    showtabs=false,                  
    tabsize=2
}% -- Setting up the custom style:

\lstset{style=mystyle}

\begin{document}

\title{showyourwork! for open science}

\subtitle{Test cases}

\author{Giovanni Picogna}

\institute{Ludwig-Maximilians-Universit\:at M\:unchen, USM, Munich, Germany\\ \email{picogna@usm.lmu.de}}

\date{Received 14 July 2025 / Accepted 14 July 2025}

\abstract {} {We look for characteristics typical of water-megamaser galaxies
in SO 103-G035, TXS 2226-184, and IC 1481.} {We obtained long-slit optical
emission-line spectra.} {We present rotation curves, line ratios, electron
densities, temperatures. IC 1481 reveals a spectrum suggestive of a vigorous
starburst in the central kiloparsec 108 years ago.} {We do not find any hints
for outflows nor special features which could give clues to the unknown
megamaser excitation mechanism.}

\keywords{interstellar medium: jets and outflows --
  interstellar medium: molecules -- stars: pre-main-sequence}
\maketitle 

% Main body with filler text
\section{Installation}
\label{sec:intro}
\begin{itemize}
\item showyourwork! requires Python <3.11, >=3.8 in the current development version.
\item we can start by creating a conda environment with python 3.10
  \begin{lstlisting}[language=bash]
    conda create -n python310 python=3.10 anaconda
  \end{lstlisting}
\item and activate it
  \begin{lstlisting}[language=bash]
    conda activate python310
  \end{lstlisting}
\item and install the development version from github
  \begin{lstlisting}[language=bash]
    pip install git+https://github.com/showyourwork/showyourwork
  \end{lstlisting}
\end{itemize}

\section{First paper}
\begin{itemize}
\item in order to set up your first paper you need to run
  \begin{lstlisting}[language=bash]
    showyourwork setup ${GITHUB_USER}/${GITHUB-REPO}
  \end{lstlisting}
\item and create the related repo on github (follow the instruction on screen - you can add the caching support on Zenodo and the Overleaf connection also in a second step)
\item A working environment.yml for the conda environment is:
  \begin{lstlisting}[language=bash]
    channels:
      - conda-forge
      - defaults
    dependencies:
      - numpy=1.21
      - pip=25.1.1
      - python=3.10
      - pip:
        - matplotlib==3.10
  \end{lstlisting}
\item you can then build first your paper locally
  \begin{lstlisting}[language=bash]
    showyourwork build
  \end{lstlisting} 
\item and check the resulting ms.pdf file in the root directory for the final result.
\end{itemize}
\subsection{Building on github}
\begin{itemize}
\item In order to build it on github thorugh github actions you need first to include/substitute in the files \texttt{build.yml} and \texttt{build-pull-request.yml} under the directory \texttt{.github/workflows} the following lines:
  \begin{lstlisting}[language=xml]
    - name: Build the article PDF
      id: build
      with:
        showyourwork-spec: git+https://github.com/showyourwork/showyourwork
      uses: showyourwork/showyourwork-action@main
      env:
        SANDBOX_TOKEN: ${{ secrets.SANDBOX_TOKEN }}
        OVERLEAF_TOKEN: ${{ secrets.OVERLEAF_TOKEN }}
  \end{lstlisting}
\item add Figure script
\item talk about snakemake and intermediate outputs
\item add static figure
\item setup Zenodo integration
\end{itemize}

\begin{figure}
    \script{random_numbers.py}
    \begin{centering}
        \includegraphics[width=\linewidth]{figures/random_numbers.pdf}
        \caption{
            Plot showing a bunch of random numbers.
        }
        \label{fig:random_numbers}
    \end{centering}
\end{figure}

\begin{figure}
    \begin{centering}
        \includegraphics[width=\linewidth]{figures/usmlogo.pdf}
        \caption{
            USM Logo.
        }
        \label{fig:usm_logo}
    \end{centering}
\end{figure}

\begin{figure}
    \script{figure.py}
    \begin{centering}
        \includegraphics[width=\linewidth]{figures/figure.pdf}
        \caption{
            Plot showing the result of a DustPy simulation.
        }
        \label{fig:simulation_dustpy}
    \end{centering}
\end{figure}

Nam dui ligula, fringilla a, euismod sodales, sollici- tudin vel, wisi.
Morbi auctor lorem non justo, nam lacus libero, pretium at, lobortis vitae.
Donec aliquet, tortor sed accumsan bibendum, erat ligula aliquet magna.
Morbi ac orci et nisl hendrerit mollis, suspendisse ut massa, cras nec ante.
Pellentesque a nulla cum sociis natoque penatibus et magnis dis parturient.
Aliquam tincidunt urna, nulla ullamcorper vestibulum turpis.
Pellentesque cursus luctus mauris \citep{Luger2021}.

\bibliography{bib}

\end{document}
